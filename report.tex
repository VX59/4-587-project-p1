
\documentclass[manuscript,screen,review]{acmart}

\settopmatter{printacmref=false} %Removes the ACM reference format (which includes DOI)
\renewcommand\footnotetextcopyrightpermission[1]{} %Removes the footnote with the conference info


%%
%% end of the preamble, start of the body of the document source.
\begin{document}

%%
%% The "title" command has an optional parameter,
%% allowing the author to define a "short title" to be used in page headers.
\title{Exploratory Data Analysis}

%%
%% The "author" command and its associated commands are used to define
%% the authors and their affiliations.
%% Of note is the shared affiliation of the first two authors, and the
%% "authornote" and "authornotemark" commands
%% used to denote shared contribution to the research.
\author{Harvey Kwong}
\email{harveykw@buffalo.edu}
\author{Jacob DeRosa}
\email{jderosa3@buffalo.edu}
\affiliation{%
  \institution{University at Buffalo}
  \city{Buffalo}
  \state{New York}
  \country{USA}
}


\maketitle

\section{Introduction and Problem Statement}

    The rise of antibiotic resistant infections poses a significant challenge to global public health, 
and gonorrhoea is one of the most concerning pathogens due to its rapidly increasing resistance to 
available treatments and its difficulty to be detected. This project focuses on predicting antibiotic resistance 
in Neisseria gonorrhoeae, the bacteria responsible for gonorrhoea, using subsets of its DNA sequences as predictive features. 
The primary objective of this project is to explore the relationship between bacterial DNA segments and 
resistance patterns, and to identify key genetic markers that could possibly predict resistance. By performing data 
cleaning and exploratory data analysis on a dataset containing DNA sequences and resistance outcomes, 
we aim to address the following question: Which DNA segments are most associated with antibiotic resistance? 

    This project’s contribution is very important as it helps to clarify biological data, making it easier for 
future studies to focus on effective solutions. By gaining a deeper understanding of the data and identifying 
key variables, this project serves as a step towards more advanced research on antibiotic resistance.

\section{Data Sources}

    We utilized the "Predicting antibiotic resistance in gonorrhoea" dataset from Kaggle: \\
    (https://www.kaggle.com/datasets/nwheeler443/gono-unitigs/data). \\


\section{Data Cleaning/Processing}



\section{Exploratory Data Analysis}


\section{Title Information}


\end{document}

